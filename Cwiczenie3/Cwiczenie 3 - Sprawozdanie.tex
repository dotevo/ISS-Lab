\documentclass[a4paper,10pt]{article}
\usepackage[utf8]{inputenc}
\usepackage{polski}
\usepackage{graphicx}
\usepackage{listings}
\usepackage[usenames,dvipsnames]{color}
\addtolength{\hoffset}{-1cm}
\addtolength{\voffset}{-2cm}
\addtolength{\textwidth}{2cm}
\addtolength{\textheight}{3cm}
\usepackage{setspace}
\usepackage{indentfirst}
\usepackage{graphicx}
\lstset{
    language=Matlab,
    basicstyle=\scriptsize,
    aboveskip={1.5\baselineskip},
    columns=fixed,
    showstringspaces=false,
    extendedchars=true,
    breaklines=true,
    tabsize=4,
    prebreak = \raisebox{0ex}[0ex][0ex]{\ensuremath{\hookleftarrow}},
    frame=single,
    showtabs=false,
    showspaces=false,
    showstringspaces=false,
    identifierstyle=\ttfamily,
    keywordstyle=\color[rgb]{0,0,1},
    commentstyle=\color[rgb]{0.133,0.545,0.133},
    stringstyle=\color[rgb]{0.627,0.126,0.941},
    numbers=left,
    numberstyle=\tiny,
    stepnumber=1,
    numbersep=5pt,
    captionpos=b,
    escapeinside={\%*}{*)}
}

\def\figurename{Rys.}
\def\lstlistingname{Fun.}

\title{Informatyczne Systemy Sterowania \\ \large Ćwiczenie 3: Regulacja dwu- i trójpołożeniowa}

\author{Adam Jordanek 168139, Tomasz Klimek 168092}

\begin{document}
\maketitle

\section{Wstęp}\label{sec:wstęp}
\subsection{Cel ćwiczenia}
Celem tego ćwiczenia jest symulacja działania systemu regulacji z przekaźnikami dwu- i trójpołożeniowymi. Ćwiczenie umożliwić zapoznanie się z nieliniowymi algorytmami sterowania (przełącznikami dwu- i trójpołożeniowymi) oraz zapoznanie się ze środowiskiem Simulink oraz Matlab w zakresie nieliniowych algorytmów sterowania.
%Czy to jest dobrze?

\subsection{Plan badań} 
\begin{enumerate}
	\item Symulacja systemu regulacji. Dobór parametrów regulatora. \newline
	%czy to jest poprawne gramatycznie? :P
	 \small {W trakcie realizacji zadania należy zasymulować działanie systemu regulacji pracującego z regulatorem oraz należy zbadać wpływ wartości parametrów przekaźników na przebieg błędu regulacji}
	\begin{enumerate}
		\item Regulator dwupołożeniowy bez histerezy.
		\item Regulator dwupołożeniowy z histerezą.
		\item Regulator trójpołożeniowy bez histerezy.
		\item Regulator trójpołożeniowy z histerezą.
 	\end{enumerate}
	\item Zastosowanie członów korekcyjnych.
	\begin{enumerate}
		\item Modyfikacja systemów z zadania 1 o korekcję w postaci członu liniowego o transmitancji: 
			\begin{equation} \label{eqn:czlonKorekcyjny}
				K_{k}(s) = {{k_{k}} \over {T_{k}s+1}}
			\end{equation}
		\item Doświadczalny dobor parametrów członu korekcyjnego.
	\end{enumerate}
\end{enumerate}

\subsection{Podział zadań. } 
\begin{enumerate}
		\item Jordanek Adam - napisanie funkcji, opracowanie zadania 2, współpraca przy pozostałej części sprawozdania.
		\item Klimek Tomasz - budowa modeli systemów, opracowanie zadania 1, współpraca przy pozostałej części sprawozdania.
\end{enumerate}

\newpage
\section{Realizacja planu i wyniki}
%--------------------------------------------------------------------------------------------------------------------------------
%ZADANIE 1
%--------------------------------------------------------------------------------------------------------------------------------
\subsection{Symulacja systemu regulacji. Dobór parametrów regulatora.}\label{sec:zad1}

System regulacji będziemy symulować przy użyciu programu Simulink będącego częścią pakietu M\small ATLAB.\normalsize

\subsubsection{Regulator dwupołożeniowy bez histerezy}\label{sec:r2bh}
%--------------------------------------------------------------------------------------------------------------------------------

Schemat systemu służącego nam do symulacji regulatora dwupołożeniowego przedstawiony został na poniższym rysunku.

\begin{figure}[!h]
    \centering
	\includegraphics[width=120mm]{CW3-schemat-2.PNG}
	\caption{Schemat symulacji regulatora dwupołożeniowego.}
    \label{fig:Rysunek}
\end{figure}

Aby zasymulować regulator dwupołożeniowy bez histerezy musimy ustawić parametry 'On' i 'Off' obiektu 'Relay1' na wartość 0. \\
Za przeprowadzenie testu odpowiedzialna jest poniższa funkcja.

\begin{lstlisting}[caption=Funkcja testująca regulator dwupołożeniowy bez histerezy.]
function testDwupoziomowy1(step, flag)
load_system('dwupoziomowy.mdl');
hold on;

set_param('dwupoziomowy/Relay1', 'On_switch_value', num2str(0));
set_param('dwupoziomowy/Relay1', 'Off_switch_value', num2str(0));

sim('dwupoziomowy.mdl');
figure(1);
uwy= u.signals.values;    
plot(tout, uwy, '-k');

figure(2);

if flag
    hold on;
    plot([0 20], [0 0], '-r');
end

ewy= e.signals.values;    
plot(tout, ewy, '-k');
end
\end{lstlisting}

Po wywołaniu tej funkcji otrzymaliśmy poniższe wykresy.

\begin{figure}[!h]
    \centering
	\includegraphics[width=120mm]{CW3-dwupolozeniowy-e-a0.png}
	\includegraphics[width=120mm]{CW3-dwupolozeniowy-u-a0.png}
	\caption{Wykresy $\varepsilon(t)$, oraz $u(t)$ dla regulatora dwupołożeniowego bez histerezy.}
    \label{fig:Rysunek}
\end{figure}

\newpage Widzimy z nich, że wykres błędu oscyluje z czasem coraz bliżej zera, więc wartość na wyjściu jest blisko pożądanej, lecz z czasem rośnie również częstotliwość przełączania regulatora, przez co znacznie rośnie jego zużycie i maleje żywotność.

\subsubsection{Regulator dwupołożeniowy z histerezą}\label{sec:r2h}
%--------------------------------------------------------------------------------------------------------------------------------

Histereza dwupoziomowa sprawia, że przełączanie stanu z wysokiego na niski nie zachodzi przy tej samej wartości co przełączanie stanu z niskiego na wysoki. Jej działanie przedstawione zostało na poniższym obrazku.

\begin{figure}[!h]
    \centering
	\includegraphics[width=120mm]{CW3-histereza-dwupoziomowa.png}
	\caption{Jak działa histereza dwupoziomowa.}
    \label{fig:Rysunek}
\end{figure}

Symulacje regulatora dwupoziomowego z histerezą przeprowadzimy na tym samym modelu, który posłużył nam w zadaniu \ref{sec:r2bh}, jednak w obiekcie 'Relay1' parametry 'On' i 'Off' będą różne, dzięki czemu ustawimy żądaną histerezę.\newpage
Za przeprowadzenie testu odpowiedzialna jest poniższa funkcja.

\begin{lstlisting}[caption=Funkcja testująca regulator dwupołożeniowy z histerezą.]
function testDwupoziomowy2(step, stop, flag)
load_system('dwupoziomowy.mdl');
hold on;
i=1;
color = char('y', 'k', 'b', 'g', 'r', 'm');

while (step*i <= stop)
    set_param('dwupoziomowy/Relay1', 'On_switch_value', num2str(step*i));
    set_param('dwupoziomowy/Relay1', 'Off_switch_value', num2str(-step*i));

    sim('dwupoziomowy.mdl');
    figure(1);
    uwy= u.signals.values;    
    plot(tout, uwy, 'Color', color(mod(i,6)+1));

    figure(2);
    ewy= e.signals.values;    
    plot(tout, ewy, 'Color', color(mod(i,6)+1));

    if flag
        hold on;
        plot([0 30], [0 0], '-r');
        plot([0 30], [step*i step*i], strcat('--', color(mod(i,6)+1)));
        plot([0 30], [-step*i -step*i], strcat('--', color(mod(i,6)+1)));
    end

    i=i+1;
    hold all;
end
end
\end{lstlisting}

Po wywołaniu powyższej funkcji dla wartości (0.5, 0.6, true) otrzymaliśmy poniższe wykresy.

\begin{figure}[!h]
    \centering
	\includegraphics[width=120mm]{CW3-dwupolozeniowy-e-a05.png}
	\includegraphics[width=120mm]{CW3-dwupolozeniowy-u-a05.png}
	\caption{Wykresy $\varepsilon(t)$, oraz $u(t)$ dla regulatora dwupołożeniowego z histerezą.}
    \label{fig:Rysunek}
\end{figure}

Widzimy na nich, że amplituda oscylacji błędu wokół zera z czasem stabilizuje się, a liczba przełączeń znacznie maleje w stosunku do regulatora dwupołożeniowego bez histerezy, a więc i wydłuża się jego żywotnosć. \\
Aby zbadać wpływ zakresu histerezy na wykres $\varepsilon(t)$ wywołamy funkcję dla wartości (0.5, 2, false), w wyniku czego otrzymamy poniższe wykresy.

\begin{figure}[!h]
    \centering
	\includegraphics[width=120mm]{CW3-dwupolozeniowy-e.png}
	\caption{Wykres $\varepsilon(t)$ dla regulatora dwupołożeniowego z histerezą przy rosnącym $a$.}
    \label{fig:Rysunek}
\end{figure}

\newpage Widzimy na nich, że w miarę zwiększania zakresu histerezy rośnie amplituda oscylacji wykresu błędu wokół zera, oraz maleje częstotliwość przełączeń regulatora.

\subsubsection{Regulator trójpołożeniowy bez histerezy}\label{sec:r3bh}
%--------------------------------------------------------------------------------------------------------------------------------

\normalsize Schemat systemu służącego nam do symulacji regulatora trójpołożeniowego przedstawiony został na poniższym rysunku.

\begin{figure}[!h]
    \centering
	\includegraphics[width=120mm]{CW3-schemat-3.png}
	\caption{Schemat symulacji regulatora trójpołożeniowego.}
    \label{fig:Rysunek}
\end{figure}

Aby zasymulować regulator trójpołożeniowy bez histerezy musimy ustawić parametry 'On' i 'Off' obiektów 'Relay1' i 'Relay2' na wartość 0.
\newpage Za przeprowadzenie testu odpowiedzialna jest poniższa funkcja.

\begin{lstlisting}[caption=Funkcja testująca regulator trójpołożeniowy bez histerezy.]
function testTrojpoziomowy(n, step, stop, flag)
load_system('trojpoziomowy.mdl');
hold on;
set_param('trojpoziomowy/Relay1', 'Off_switch_value', num2str(n));
set_param('trojpoziomowy/Relay1', 'On_switch_value', num2str(n));
set_param('trojpoziomowy/Relay2', 'Off_switch_value', num2str(-n));
set_param('trojpoziomowy/Relay2', 'On_switch_value', num2str(-n));

sim('trojpoziomowy.mdl');

figure(1);
uwy= u.signals.values;    
plot(tout, uwy, '-k');

figure(2);

if flag
    hold on;
    plot([0 30], [n n], '-r');
    plot([0 30], [-n -n], '-r');
end

ewy= e.signals.values;    
plot(tout, ewy, '-k');
end
\end{lstlisting}

Po wywołaniu tej funkcji otrzymaliśmy poniższe wykresy.

\begin{figure}[!h]
    \centering
	\includegraphics[width=120mm]{CW3-trojpolozeniowy-e-n015-a0.png}
	\includegraphics[width=120mm]{CW3-trojpolozeniowy-u-n015-a0.png}
	\caption{Wykresy $\varepsilon(t)$, oraz $u(t)$ dla regulatora trójpołożeniowego bez histerezy.}
    \label{fig:Rysunek}
\end{figure}

\newpage Widzimy z nich, że wykres błędu oscyluje z czasem coraz bliżej zera, więc wartość na wyjściu jest blisko pożądanej, lecz z czasem rośnie również częstotliwość przełączania regulatora, przez co znacznie rośnie jego zużycie i maleje żywotność. \\
Widzimy również, że regulator trójpołożeniowy w stan niski wszedł tylko raz, a później przełączał się tylko na stan wysoki i zerowy. Dzieje się tak, ponieważ przełączenia regulatora z czasem są coraz krótsze, więc tempo malenia wartości błędu nie ma szans wzrosnąć na tyle, aby 'przelecieć' siłą bezwładności systemu do wartości $-n$, gdzie nastąpiłoby przełączenie na stan niski. 

\subsubsection{Regulator trójpołożeniowy z histerezą}\label{sec:r3h}
%--------------------------------------------------------------------------------------------------------------------------------

Histereza trójpoziomowa to nic innego niż połączenie dwóch histerez dwupoziomowych. Jej działanie przedstawione zostało na poniższym obrazku.

\begin{figure}[!h]
    \centering
	\includegraphics[width=120mm]{CW3-histereza-trojpoziomowa.png}
	\caption{Jak działa histereza trójpoziomowa.}
    \label{fig:Rysunek}
\end{figure}

Symulacje regulatora trójpołożeniowego z histerezą przeprowadzimy na tym samym modelu, który posłużył nam w zadaniu \ref{sec:r3bh}, jednak w obiektach 'Relay1' i 'Relay2' parametry 'On' i 'Off' będą różne, dzięki czemu ustawimy żądaną histerezę.
Za przeprowadzenie testu odpowiedzialna jest poniższa funkcja.

\begin{lstlisting}[caption=Funkcja testująca regulator trójpołożeniowy z histerezą.]
function testTrojpoziomowy(n, step, stop, flag)
load_system('trojpoziomowy.mdl');
hold on; i=1;
color = char('y', 'k', 'b', 'r', 'g', 'm');
while (step*i <= stop)
    set_param('trojpoziomowy/Relay1', 'Off_switch_value', num2str(n-step*i));
    set_param('trojpoziomowy/Relay1', 'On_switch_value', num2str(n+step*i));
    set_param('trojpoziomowy/Relay2', 'Off_switch_value', num2str(-(n+step*i)));
    set_param('trojpoziomowy/Relay2', 'On_switch_value', num2str(-(n-step*i)));
    sim('trojpoziomowy.mdl');
    figure(1);
    uwy= u.signals.values;    
    plot(tout, uwy, 'Color', color(mod(i,6)+1));
    figure(2);
    ewy= e.signals.values;    
    plot(tout, ewy, 'Color', color(mod(i,6)+1));
    if flag
        plot([0 30], [n n], '-r');
        plot([0 30], [-n -n], '-r');
        plot([0 30], [n+step*i n+step*i], strcat('--', color(mod(i,6)+1)));
        plot([0 30], [n-step*i n-step*i], strcat('--', color(mod(i,6)+1)));
        plot([0 30], [-(n-step*i) -(n-step*i)], strcat('--', color(mod(i,6)+1)));
        plot([0 30], [-(n+step*i) -(n+step*i)], strcat('--', color(mod(i,6)+1)));
    end
    i=i+1;
    hold all;
end
end
\end{lstlisting}

Po wywołaniu powyższej funkcji dla wartości (0.15, 0.05, 0.06, true) otrzymaliśmy poniższe wykresy.

\begin{figure}[!h]
    \centering
	\includegraphics[width=120mm]{CW3-trojpolozeniowy-e-n015-a005.png}
	\includegraphics[width=120mm]{CW3-trojpolozeniowy-u-n015-a005.png}
	\caption{Wykresy $\varepsilon(t)$, oraz $u(t)$ dla regulatora trójpołożeniowego z histerezą.}
    \label{fig:Rysunek}
\end{figure}

Widzimy na nich, że amplituda oscylacji błędu wokół zera z czasem stabilizuje się, a liczba przełączeń znacznie maleje w stosunku do regulatora trójpołożeniowego bez histerezy, a więc i wydłuża się jego żywotnosć. \\
Aby zbadać wpływ zakresu histerezy na wykres $\varepsilon(t)$ wywołamy funkcję dla wartości (0.15, 0.05, 0.16, false), w wyniku czego otrzymamy poniższe wykresy.

\begin{figure}[!h]
    \centering
	\includegraphics[width=120mm]{CW3-trojpolozeniowy-e-n015.png}
	\caption{Wykres $\varepsilon(t)$ dla regulatora trójpołożeniowego z histerezą przy rosnącym $a$.}
    \label{fig:Rysunek}
\end{figure}

\newpage Widzimy na nich, że w miarę zwiększania zakresu histerezy rośnie amplituda oscylacji wokół zera wykresu błędu, oraz maleje częstotliwość przełączeń regulatora.

%---------------------------------------------------------------------------------------------------------------------
%ZADANIE 2
%---------------------------------------------------------------------------------------------------------------------

\subsection{Zastosowanie członów korekcyjnych.}\label{sec:zad2}
Podstawową zaletą regulacji dwupołożeniowej jest prostota realizacji. Niestety, cecha ta jest
okupiona pogorszeniem jakości parametrów regulacji w porównaniu regulacją ciągłą.
Jedną z możliwości poprawienia jakości regulacji jest zastosowanie układu z korekcją.
\newline
Człon korekcyjny posiada następującą transmitancję:

\begin{equation} \label{eqn:trans_czl_kor}
	G_w={{k} \over {T_k s+1}}
\end{equation}

\subsubsection{Modyfikacja systemów z zadania 1}\label{sec:zad2_1}
\begin{figure}[!h]
    \centering
	\includegraphics[width=120mm]{CW3-schemat-2k.png}
	\caption{Schemat regulatora dwupolożeniowego z korekcją.}
    \label{fig:Rysunek}
\end{figure}

\begin{figure}[!h]
    \centering
	\includegraphics[width=120mm]{CW3-schemat-3k.PNG}
	\caption{Schemat regulatora trojpolożeniowego z korekcją.}
    \label{fig:Rysunek}
\end{figure}
Człon korekcyjny włączony w obwód ujemnego sprzężenia zwrotnego przekaźnika wprowadza modyfikację sygnału
błędu e(t), co powoduje częstsze niż bez korekcji przełączanie przekaźnika. Korekcja jest skuteczna
wówczas gdy, prędkość narastania sygnału z korektora w jest większa od prędkości zmian wejścia (dobór kw i Tw). Zsumowanie sygnałów sprzężenia zwrotnego z obiektu i korektora oraz odjęcie ich sumy od wartości zadanej daje zastępczy sygnał błędu, którego prędkość i sposób narastania jest określona głównie przez sygnał. 

\newpage

\subsubsection{Doświadczalny dobor parametrów członu korekcyjnego.}\label{sec:zad2_2}
\begin{enumerate}
		\item Regulator dwupołożeniowy.
		
		Do sprawdzania regulatorów wykorzystaliśmy następujące funkcje testujące.
		
\begin{lstlisting}[caption=Funkcja testująca regulator dwupoziomowy z korekcją zmiana T.]
function testDwupoziomowyKorekcjaT(start,step, stop)
load_system('dwupoziomowyKorekcja.mdl');
hold on;
i=1;
color = char('y', 'k', 'b', 'g', 'r', 'm');
legendtext{1}='';
legendtext1{1}='';
while (start+step*i <= stop)
        set_param('dwupoziomowyKorekcja/Transfer Fcn2', 'Denominator', strcat('[',num2str(start+step*i),' 1]'));
        sim('dwupoziomowyKorekcja.mdl');
        figure(1);
        uwy= u.signals.values;    
        plot(tout, uwy, 'Color', color(mod(i,6)+1));        
        legendtext{i}= strcat('Tk=' , num2str(start+step*i));
        legend(legendtext1);
        figure(2);
        ewy= e.signals.values;    
        plot(tout, ewy, 'Color', color(mod(i,6)+1));        
        legendtext1{i}= strcat('Tk=' , num2str(start+step*i));
        legend(legendtext1);
        i=i+1;
        hold all;
end
end
\end{lstlisting}

Analogicznie do testowania K i C

\begin{lstlisting}[caption=set param dla K.]
set_param('dwupoziomowyKorekcja/Transfer Fcn2', 'Numerator', num2str(start+step*i));
\end{lstlisting}

\begin{lstlisting}[caption=set param dla C.]
set_param('dwupoziomowyKorekcja/Constant', 'value', num2str(start+step*i));
\end{lstlisting}

\newpage

Na początek testowany był układ z histerezą. Zadanie polegało on doświadczalnym dobraniu parametrów układu korekcyjnego.
Parametrami nas interesującymi jest Tk, K i C gdzie C to wartość stała (blok Constant).

\begin{figure}[!h]
    \centering
	\includegraphics[width=120mm]{CW3-korekcja-dwupolozeniowy-e_Tk.png}
	\caption{Zmiana Tk przy stałym k=1 i C=0}
    \label{fig:Rysunek}
\end{figure}

Z Wyresu można zauważyć, że dla wartości k=1, wartość optymalna Tk=1.5 ze względu, że nie występują duże wachania wartości.

\begin{figure}[!h]
    \centering
	\includegraphics[width=120mm]{CW3-korekcja-dwupolozeniowy-e_k.png}
	\caption{Zmiana k przy stałym Tk=1.5 i C=0}
    \label{fig:Rysunek}
\end{figure}
Gdy Tk=1 najbardziej optymalne k=0.5 ze względu, że osiąga wartość 0.
\begin{figure}[!h]
    \centering
	\includegraphics[width=120mm]{CW3-korekcja-dwupolozeniowy-e_C_k1T1.png}
	\caption{Zmiana C przy stałym k=0.5 i Tk=1.5}
    \label{fig:Rysunek}
\end{figure}
Z wykresu można zauważyć, że najlepiej układ się zachowuje kiedy C jest równe 0.

\newpage
Widać, że układ najlepiej zachoduje się gdy K=0.5 i Tk=1.
Da takich parametrów wykres dla a=0.5 wygląda:
\begin{figure}[!h]
    \centering
	\includegraphics[width=120mm]{CW3-korekcja-dwupolozeniowy-e_a.png}
	\includegraphics[width=120mm]{CW3-korekcja-dwupolozeniowy-u_a.png}
	\caption{Przy zmianie a}
    \label{fig:Rysunek}
\end{figure}


\newpage

Dla układu bez histerezy otrzymaliśmy bardzo podobne wyniki:
\begin{figure}[!h]
    \centering
	\includegraphics[width=120mm]{CW3-korekcja-dwupolozeniowyBH-e_Tk.png}
	\caption{Zmiana Tk przy stałym k=0.5 i C=0}
    \label{fig:Rysunek}
\end{figure}

\begin{figure}[!h]
    \centering
	\includegraphics[width=120mm]{CW3-korekcja-dwupolozeniowyBH-e_k.png}
	\caption{Zmiana K}
    \label{fig:Rysunek}
\end{figure}


\newpage


		\item Regulator trójpołożeniowy
Do testowania regulatora wykorzystaliśmy funkcje z regulatora dwupołożeniowego.	Całe ćwiczenie zostało wykonane w analogiszny sposób do poprzedniego.
		
\begin{figure}[!h]
    \centering
	\includegraphics[width=120mm]{CW3-korekcja-trojpolozeniowy-e_Tk.png}
	\caption{Przy zmianie Tk}
    \label{fig:Rysunek}
\end{figure}

\begin{figure}[!h]
    \centering
	\includegraphics[width=120mm]{CW3-korekcja-trojpolozeniowy-e_k.png}
	\caption{Przy zmianie K}
    \label{fig:Rysunek}
\end{figure}

\begin{figure}[!h]
    \centering
	\includegraphics[width=120mm]{CW3-korekcja-trojpolozeniowy-e_C.png}
	\caption{Przy zmianie C}
    \label{fig:Rysunek}
\end{figure}
Z wykresów można wywnioskować, że stała nie ma wpływu na regulację.
\newline
Wybrane parametry to:
Tk=20
K=0.1
C=-2
\newpage
Przy zmianie 'a' wykres wyglada:
\begin{figure}[!h]
    \centering
	\includegraphics[width=120mm]{CW3-korekcja-trojpolozeniowyBH-e_a.png}
	\caption{Przy zmianie a}
    \label{fig:Rysunek}
\end{figure}

\newpage
Dla układu bez histerezy wykonaliśmy takie same pomiary.
\begin{figure}[!h]
    \centering
	\includegraphics[width=120mm]{CW3-korekcja-trojpolozeniowyBH-e_Tk.png}
	\caption{Przy zmianie Tk bez histerezy}
    \label{fig:Rysunek}
\end{figure}

\begin{figure}[!h]
    \centering
	\includegraphics[width=120mm]{CW3-korekcja-trojpolozeniowyBH-e_k.png}
	\caption{Przy zmianie K bez histerezy}
    \label{fig:Rysunek}
\end{figure}

\end{enumerate}
\newpage
\newpage
\section{Wnioski.}\label{sec:wnioski}
Histereza pozwala nam na zmieniszenie zużycia sprzętu przez zmniejszenie częstotliwości przełączeń między stanami, kosztem odchyłów od pożądanych wartości. Wybór zakresu histerezy zależy od systemu w którym działa regulator oraz od tego jakiej dokładności wyjściowej wymaga.

Regulator trójpołożeniowy pozwa na określenie 3 stanów, dzięki którym uzyskujemy większą kontrolę nad działaniem systemu.

Człon korekcyjny pozwala dostosować system do naszych potrzeb, pozwala on na zmianę częstotliwości przęłączania.

Koszt wydruku w ksero wzrósł dlatego dobrym rozwiązaniem będzie przerzucenie się na system elektroniczny. 
Pozwoli to na dodatkowe piwko oraz uratuje lasy deszczowe.

\end{document}